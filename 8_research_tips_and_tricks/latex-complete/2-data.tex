%!TEX root = 0-paper.tex
% The header above is for LaTeXTools, so we can build the PDF from this file
% Without it, LaTeX will not build the file because it lacks the document class
% and the rest of the header

% Let's write a table - the table environment creates a special
% box which LaTeX will place where convenient (a "float")
% The inner tabular environment provides an actual table structure
% The letter in bracket is a float specification, h stands for here
% it tells LaTeX to 
\begin{table}[h]
\centering % Always a good idea
% We create a table with three columns (left-aligned, centered, 
% right-aligned) separated with vertical lines
\begin{tabular}{|l|c|r|}
% We can add horizontal lines with the \hline command
\hline
% The & symbol separates columns, and the \\ indicates a new line
Name & Age & Profession\\
\hline
John Doe & 34 & Doctor\\
\hline
Jane Doe & 36 & Lawyer\\
\hline
Barack Obama & 57 & Unemployed\\
\hline
\end{tabular}
% Always useful to include a caption
\caption{A bad table}
% The label is one of the most useful LaTeX features
% It allows us to refer to this table in the text and let LaTeX
% deal with the numbering
\label{tab:bad}
\end{table}

% What's wrong with this table?
% a) VERTICAL LINES ARE NEVER GOOD
% b) top and bottom are not clearly delimited
% c) heading row is not different from other rows

% Now let's make a good-looking table using the booktabs package
\begin{table}[h]
\centering
\begin{tabular}{lcr}
\toprule
Name & Age & Profession\\
\midrule
John Doe & 34 & Doctor\\
Jane Doe & 36 & Lawyer\\
Barack Obama & 57 & Unemployed\\
\bottomrule
\end{tabular}
\caption{A good table}
\label{tab:good}
\end{table}

Table~\ref{tab:bad} looks much worse than Table~\ref{tab:good}.
